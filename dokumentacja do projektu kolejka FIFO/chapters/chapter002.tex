\chapter{Opis struktury projektu}
\label{cha:Opis struktury projektu}

Projekt został zaimplementowany w języku Java z wykorzystaniem bibliotek Swing i JDBC. Dane są przechowywane w relacyjnej bazie danych MySQL.
W projekcie zastosowano 7 klas oraz plik mysql-connector-j-9.3.0.jar jest to sterownik, który umożliwia połączenie programu z bazą danych MYSQL.

%---------------------------------------------------------------------------

\section{Polecenie SQL:}
\label{sec:Polecenie SQL:}

% ------------------------
\begin{lstlisting}
CREATE DATABASE IF NOT EXISTS zamowienia_db;
USE zamowienia_db;

CREATE TABLE IF NOT EXISTS zamowienia (
id INT AUTO_INCREMENT PRIMARY KEY,
imie VARCHAR(255) NOT NULL,
email VARCHAR(255) NOT NULL
);

CREATE TABLE IF NOT EXISTS produkty (
id INT AUTO_INCREMENT PRIMARY KEY,
nazwa VARCHAR(255) NOT NULL,
cena DOUBLE NOT NULL,
zamowienie_id INT,
FOREIGN KEY (zamowienie_id) REFERENCES zamowienia(id) ON DELETE CASCADE
);
\end{lstlisting}
%---------------------------------------------------------------------------


% ------------------------
\section{Zarządzanie danymi i baza danych:}
\label{sec:Zarządzanie danymi i baza danych:}

\begin{itemize}
	\item Tabela zamowienia: ID, imie, email


%
\begin{table}[H]
	\centering
	\caption{Tabela Zamówienia (Baza danych)\label{tab}}
	\renewcommand{\tabularxcolumn}[1]{m{#1}}  % Dostosowanie kolumny do zawartości
	\newcolumntype{C}{>{\centering\arraybackslash}X}
	\begin{tabularx}{\linewidth}{CCC}
		\toprule
		\textbf{Kolumna}              & \textbf{Typ}              & \textbf{Uwagi}         \\
		\midrule
		\multirow[m]{1}{*}{id} & INT                     & AUTO INCREMENT, PK              \\
		\midrule
		\multirow[m]{1}{*}{imie} & VARCHAR(255)                     & Imię i nazwisko             \\
		\midrule
		\multirow[m]{1}{*}{e-mail} & VARCHAR(255)                     & Adres e-mail              \\
		\midrule
		
	\end{tabularx}
\end{table}

% ------------------------------------------------------------------------
	\item Tabela produkty: ID, nazwa, cena, zamowienieid.

\begin{table}[H]
	\centering
	\caption{Tabela produkty (Baza danych)\label{tab1}}
	\renewcommand{\tabularxcolumn}[1]{m{#1}}  % Dostosowanie kolumny do zawartości
	\newcolumntype{C}{>{\centering\arraybackslash}X}
	\begin{tabularx}{\linewidth}{CCC}
		\toprule
		\textbf{Kolumna}              & \textbf{Typ}              & \textbf{Uwagi}         \\
		\midrule
		\multirow[m]{1}{*}{id} & INT                     & AUTOINCREMENT, PK              \\
		\midrule
		\multirow[m]{1}{*}{nazwa} & VARCHAR(255)                     & Nazwa produktu             \\
		\midrule
		\multirow[m]{1}{*}{cena} & DOUBLE                     & Cena jednostkowa              \\
		\midrule
		\multirow[m]{1}{*}{zamowienie id} & INT                     & FK -> zamowienia(id)              \\
		\midrule
		
	\end{tabularx}
\end{table}
\end{itemize}
%---------------------------------------------------------------------------

\section{Dane Techniczne:}
\label{sec:Narzędzia:}
Minimalne wymagania sprzętowe:
\begin{itemize}
	\item Procesor: 1.5 GHz,
	\item RAM: 2 GB
	\item Dysk: 100 MB wolnego miejsca
	\item Zainstalowana Java 8+
	\item System operacyjny: Windows 10+ lub Linux
\end{itemize}
Narzędzia:
\begin{itemize}
      \item IntelliJ IDEA.
      \item MySQL Server
      \item JDBC
      \item Java SE
      \item Swing GUI Designer
\end{itemize}

%---------------------------------------------------------------------------


\section{Działanie}


      \begin{enumerate}
            \item Użytkownik uruchamia aplikację (Main –> GUI).
            \item GUI pokazuje okno z przyciskami: "Dodaj", "Pobierz", "Poka˙z klientów".
            \item Użytkownik dodaje zamówienie:
            \begin{itemize}
            	\item Podaje dane klienta i list˛e produktów przez JOptionPane.
            	\item Zamówienie trafia do kolejki FIFO (KolejkaZamówień).
            	\item Dane są zapisywane do bazy danych (MenagerBazyDanych).
            \end{itemize} 
            \item Użytkownik może zrealizować zamówienie z kolejki.
            \item Może również podejrzeć wszystkie zapisane zamówienia z bazy.
		\end{enumerate}


