\chapter{Wprowadzenie}
\label{cha:wprowadzenie}

Kolejka FIFO (First In, First Out) to struktura danych, w której elementy sa˛ przetwarzane w takiej
kolejności, w jakiej zostały dodane. Oznacza to, że pierwszy element dodany do kolejki zostanie również
jako pierwszy z niej usunięty — dokładnie jak kolejka ludzi w sklepie. Celem projektu jest stworzenie graficznej aplikacji komputerowej do zarządzania zamówieniami klientów przy użyciu struktury kolejki FIFO. Aplikacja umożliwia dodawanie zamówień, ich realizację oraz przeglądanie zapisanych danych klientów. Dane zapisywane są w bazie danych MySQL. Projekt łączy technologię Swing (interfejs graficzny) z JDBC (połączenie z bazą danych). Kluczowe cechy FIFO:
\begin{itemize}
	\item Dodawanie — nowy element trafia na koniec kolejki.
	\item Usuwanie — element usuwany jest z początku kolejki.
	\item Zastosowanie — systemy kolejkowe, buforowanie, drukarki, planowanie zadań (np. w systemach operacyjnych), sieci komputerowe itp.
\end{itemize}

%---------------------------------------------------------------------------


\section{Opis założeń projektu}
\label{sec:Opis założeń projektu}

Celem poniższej pracy jest symulacja kolejki FIFO w języku Java w kontekście zamówień w sklepie. Podstawowym założeniem jest, aby obsługiwać zamówienia w kolejności, w jakiej zostały złożone. Oznacza to, że zamówienie, które pojawiło się jako pierwsze, będzie realizowane jako pierwsze, a dopiero w następnej kolejności te złożone później. Problemem rozwiązywanym przez aplikację jest brak uporządkowanego systemu obsługi zamówień, co często prowadzi do chaosu, opóźnień oraz braku przejrzystości w kolejności realizacji zleceń. Korzyści kolejki:
\begin{itemize}
	\item Zmniejszenie ryzyka pomyłek
	\item Zadowolenie klientów
	\item Sprawiedliwość
	\item Prostota i przejrzystość
	\item Zachowanie kolejności obsługi
\end{itemize} 
Głównym źródłem problemu jest brak informatyzacji w małych podmiotach, które nie posiadają systemu ERP ani dedykowanego narzędzia do zarządzania kolejkami. Problem ten jest ważny, ponieważ wpływa na jakość obsługi klienta, czas realizacji zamówień i efektywność operacyjną.

% ------------------------
Wymagania funkcjonalne:
\begin{itemize}
	\item Dodawanie nowego zamówienia do kolejki.
	\item Pobieranie zamówienia (realizacja) z kolejki.
	\item Zapisywanie zamówienia do bazy danych.
	\item Wyświetlanie listy klientów i ich zamówień z bazy danych
	\item Obsługa wielu produktów w jednym zamówieniu.
	\item Interfejs graficzny pozwalający na interakcję z użytkownikiem.
\end{itemize}

% ------------------------
Wymagania niefunkcjonalne:
\begin{itemize}
	\item Intuicyjny interfejs graficzny
	\item Działanie aplikacji offline z możliwością zapisu do lokalnej bazy danych
	\item Niska awaryjność i obsługa błędów
	\item Szybkość działania nawet przy większej liczbie zamówień
	\item Minimalne wymagania sprzętowe.
\end{itemize}
%---------------------------------------------------------------------------
\section{Zawartosć pracy}
\label{sec:Zawartosć pracy:}
W projekcie zawarto 7 klas o następującej hierarchi:
\begin{itemize}
	\item GUI – klasa odpowiedzialna za interfejs graficzny (JFrame). Obsługuje przyciski: dodaj, pobierz, pokaż klientów.
	\item KolejkaZamówień – logika kolejki FIFO, oparta na LinkedList. Metody: dodajZamowienie(), pobierzZamowienie(), isEmpty().
	\item Klient, Produkt, Zamówienie – klasy modelowe reprezentujące dane
	\item MenagerBazyDanych – obsługuje połączenie z MySQL oraz operacje: dodajZamowienie(), pobierzZamowienia().
\end{itemize}
W Projekcie użyto także Bazy danych MYSQL
